\documentclass{report}
\usepackage[showframe=false]{geometry}
\usepackage{titlesec}
\usepackage{amsmath}
\usepackage{graphicx}
\usepackage{tikz,pgfplots}
\usepackage{multicol}

\usepgfplotslibrary{fillbetween}
\usepgfplotslibrary{groupplots}
\usetikzlibrary{arrows,shapes.multipart,calc}

\pagenumbering{gobble}

\geometry{tmargin=60pt,bmargin=90pt,lmargin=90pt,
rmargin=90pt}

\titleformat{\chapter}{\normalfont\huge}{\thechapter.}{20pt}{\huge}
\titlespacing*{\chapter} {0pt}{0pt}{10pt}

\definecolor{cadmiumgreen}{rgb}{0.0, 0.42, 0.24}

\begin{document}

\chapter{Question 1}

Consider  one auto  company that  receives  parts from  three suppliers;  assume  
50\% of  the parts from  supplier  1,  30\% from  supplier  2,  and 20\% from  supplier  3.  
The quality of  the parts could be  summarized  in  the following table based on  
historically  data.

\begin{center}
  \begin{tabular}{ | c | c | c | } 
    \hline
      & Percentage Good Parts & Percentage Bad parts \\ 
    \hline
    Supplier 1 & 98 & 2 \\ 
    \hline
    Supplier 2 & 95 & 5 \\ 
    \hline
    Supplier 3 & 92 & 8 \\ 
    \hline
  \end{tabular}
\end{center}

Question: A bad part  broke one of  the machines  (observed), what  is  the 
probability the part  came  from  supplier  1?

\hspace{1cm}

\textbf{ANSWER:} \\

Let $A_1$ denote Supplier 1, $A_2$ denote Supplier 2, and $A_3$ denote Supplier 3

\begin{equation} \label{eq3}
  \begin{split}
    P(A_1) & = 0.50 \\
    P(A_2) & = 0.30 \\
    P(A_3) & = 0.20 \\
  \end{split}
\end{equation}

Let G denote that a part is good and B denote that a part is bad. \\

\begin{equation} \label{eq3}
  \begin{split}
    P(G|A_1) & = 0.98 \\
    P(G|A_2) & = 0.95 \\
    P(G|A_3) & = 0.92 \\
  \end{split}
\end{equation}

\begin{equation} \label{eq3}
  \begin{split}
    P(B|A_1) & = 0.02 \\
    P(B|A_2) & = 0.05 \\
    P(B|A_3) & = 0.08 \\
  \end{split}
\end{equation}

\break

Here is the Probability Tree for Three-Suppliers

\hspace{1cm}

% Set the overall layout of the tree
\tikzstyle{level 1}=[level distance=3.5cm, sibling distance=3.5cm]
\tikzstyle{level 2}=[level distance=3.5cm, sibling distance=2cm]

% Define styles for bags and leafs
\tikzstyle{bag} = [circle, minimum width=3pt, fill, inner sep=0pt]
\tikzstyle{end} = [circle, minimum width=3pt, fill, inner sep=0pt]

% The sloped option gives rotated edge labels. Personally
% I find sloped labels a bit difficult to read. Remove the sloped options
% to get horizontal labels. 
\begin{tikzpicture}[grow=right, sloped]
  \node[bag] {$\circ$}
  child {
    node[bag] {$\circ$}
    child {
      node[end, label=right:
      {$P(A_3\cap B)=P(A_3)P(B|A_3)=0.016$}] {}
      edge from parent
      node[above] {$P(B|A_3)$}
    node[below]  {\textcolor{red}{$0.08$}}
    }
    child {
      node[end, label=right:
      {$P(A_3\cap G)=P(A_3)P(G|A_3)=0.184$}] {}
      edge from parent
      node[above] {$P(G|A_3)$}
      node[below]  {\textcolor{cadmiumgreen}{$0.92$}}
    }
    edge from parent 
    node[above] {$P(A_3)$}
    node[below]  {\textcolor{blue}{$0.20$}}
  } child {
    node[bag] {$\circ$}
    child {
      node[end, label=right:
      {$P(A_2\cap B)=P(A_2)P(B|A_2)=0.015$}] {}
      edge from parent
      node[above] {$P(B|A_2)$}
      node[below]  {\textcolor{red}{$0.05$}}
    }
    child {
      node[end, label=right:
      {$P(A_2\cap G)=P(A_2)P(G|A_2)=0.285$}] {}
      edge from parent
      node[above] {$P(G|A_2)$}
      node[below]  {\textcolor{cadmiumgreen}{$0.95$}}
    }
    edge from parent         
    node[above] {$P(A_2)$}
    node[below]  {\textcolor{blue}{$0.30$}}
  } child {
    node[bag] {$\circ$}
    child {
      node[end, label=right:
      {$P(A_1\cap B)=P(A_1)P(B|A_1)=0.010$}] {}
      edge from parent
      node[above] {$P(B|A_1)$}
      node[below]  {\textcolor{red}{$0.02$}}
    }
    child {
      node[end, label=right:
      {$P(A_1\cap G)=P(A_1)P(G|A_1)=0.490$}] {}
      edge from parent
      node[above] {$P(G|A_1)$}
      node[below]  {\textcolor{cadmiumgreen}{$0.98$}}
    }
    edge from parent         
    node[above] {$P(A_1)$}
    node[below]  {\textcolor{blue}{$0.50$}}
  };
\end{tikzpicture}

\hspace{1cm}

Law of Conditional Probability gives us the following equation

\begin{equation} \label{eq3}
  P(A_1|B) = \frac{P(A_1\cap B)}{P(B)}
\end{equation}

We know from are probability tree that

\begin{equation} \label{eq3}
P(A_1\cap B)=P(A_1)P(B|A_1)=0.010
\end{equation}

We also know that

\begin{equation} \label{eq3}
  P(B) = P(A_1\cap B) + P(A_2\cap B) + P(A_3\cap B) = P(A_1)P(B|A_1) + P(A_2)P(B|A_2) + P(A_3)P(B|A_3)
\end{equation}

Combining are equations togther we obtain Bayes' Theorem

\begin{equation}
  \begin{split}
    & 1 \leq k \leq n \\
    P(B_k|A) & = \frac{P(A|B_k)P(B_k)}{\sum_{i=1}^{n}P(A|B_i)P(B_i)}
  \end{split}
\end{equation}

\begin{equation}
  \begin{split}
    P(A_1|B) & = \frac{P(A_1)P(B|A_1)}{P(A_1)P(B|A_1)+P(A_2)P(B|A_2)+P(A_3)P(B|A_3)} \\
    P(A_2|B) & = \frac{P(A_2)P(B|A_2)}{P(A_1)P(B|A_1)+P(A_2)P(B|A_2)+P(A_3)P(B|A_3)} \\
    P(A_3|B) & = \frac{P(A_3)P(B|A_3)}{P(A_1)P(B|A_1)+P(A_2)P(B|A_2)+P(A_3)P(B|A_3)} \\
  \end{split}
\end{equation}

\break

Now all we have to do is plug in the numbers and solve the equations

\begin{equation}
  \begin{split}
  P(A_1|B) & = \frac{(0.50)(0.02)}{(0.50)(0.02)+(0.30)(0.05) +(0.20)(0.08)} = \frac{0.010}{0.041} = 0.2439024390 \\
  P(A_2|B) & = \frac{(0.30)(0.05)}{(0.50)(0.02)+(0.30)(0.05) +(0.20)(0.08)} = \frac{0.015}{0.041} = 0.3658536585 \\
  P(A_3|B) & = \frac{(0.20)(0.08)}{(0.50)(0.02)+(0.30)(0.05) +(0.20)(0.08)} = \frac{0.016}{0.041} = 0.3902439024 \\
  \end{split}
\end{equation}

Therefore the probability the part came from supplier 1: 0.243902439

\chapter{Question 2}

For the play  tennis  data  set shown below:

\begin{tabular}{ |c||c c c c | c | }
  \hline
  Day & Outlook & Temperature & Humidity & Wind & PlayTennis \\
  \hline
  D1  & Sunny                     & Hot                   & High                    & Weak                    & No                   \\
  D2  & Sunny                     & Hot                   & High                    & Strong                  & No                   \\
  D3  & \textcolor{red}{Overcast} & \textcolor{red}{Hot}  & \textcolor{red}{High}   & \textcolor{red}{Weak}   & \textcolor{red}{Yes} \\
  D4  & \textcolor{red}{Rain}     & \textcolor{red}{Mild} & \textcolor{red}{High}   & \textcolor{red}{Weak}   & \textcolor{red}{Yes} \\
  D5  & \textcolor{red}{Rain}     & \textcolor{red}{Cool} & \textcolor{red}{Normal} & \textcolor{red}{Weak}   & \textcolor{red}{Yes} \\
  D6  & Rain                      & Cool                  & Normal                  & Strong                  & No                   \\
  D7  & \textcolor{red}{Overcast} & \textcolor{red}{Cool} & \textcolor{red}{Normal} & \textcolor{red}{Strong} & \textcolor{red}{Yes} \\
  D8  & Sunny                     & Mild                  & High                    & Weak                    & No                   \\
  D9  & \textcolor{red}{Sunny}    & \textcolor{red}{Cool} & \textcolor{red}{Normal} & \textcolor{red}{Weak}   & \textcolor{red}{Yes} \\
  D10 & \textcolor{red}{Rain}     & \textcolor{red}{Mild} & \textcolor{red}{Normal} & \textcolor{red}{Weak}   & \textcolor{red}{Yes} \\
  D11 & \textcolor{red}{Sunny}    & \textcolor{red}{Mild} & \textcolor{red}{Normal} & \textcolor{red}{Strong} & \textcolor{red}{Yes} \\
  D12 & \textcolor{red}{Overcast} & \textcolor{red}{Mild} & \textcolor{red}{High}   & \textcolor{red}{Strong} & \textcolor{red}{Yes} \\
  D13 & \textcolor{red}{Overcast} & \textcolor{red}{Hot}  & \textcolor{red}{Normal} & \textcolor{red}{Weak}   & \textcolor{red}{Yes} \\
  D14 & Rain                      & Mild                  & High                    & Strong                  & No                   \\
  \hline
\end{tabular}

\section{Part 1}

Please use	Naïve	Bayes	to	help	make	decision	on	playing	tennis	or	not	when
Outlook	is	rain,	Temperature	is	mild,	Humidity	is	normal and	Wind	is	weak.

\hspace{1cm}

\textbf{ANSWER:} \\

\begin{tabular}{ |c|c|c|c| }
  \hline
  \textbf{OUTLOOK} & Play=Yes & Play=No & Total \\
  \hline
  Sunny            & 2/9      & 3/5     & 5/14  \\
  \hline
  Overcast         & 4/9      & 0/5     & 4/14  \\
  \hline
  Rain             & 3/9      & 2/5     & 5/14  \\
  \hline
\end{tabular}

\vspace*{1 cm}

\begin{tabular}{ |c|c|c|c| }
  \hline
  \textbf{TEMPERATURE} & Play=Yes & Play=No & Total \\
  \hline
  Hot                  & 2/9      & 2/5     & 4/14  \\
  \hline
  Mild                 & 4/9      & 2/5     & 6/14  \\
  \hline
  Cool                 & 3/9      & 1/5     & 4/14  \\
  \hline
\end{tabular}

\vspace*{1 cm}

\begin{tabular}{ |c|c|c|c| }
  \hline
  \textbf{HUMIDITY} & Play=Yes & Play=No & Total \\
  \hline
  High              & 3/9      & 4/5     & 7/14  \\
  \hline
  Normal            & 6/9      & 1/5     & 7/14  \\
  \hline
\end{tabular}

\vspace*{1 cm}

\begin{tabular}{ |c|c|c|c| }
  \hline
  \textbf{WIND} & Play=Yes & Play=No & Total \\
  \hline
  Strong        & 3/9      & 3/5     & 6/14  \\
  \hline
  Weak          & 6/9      & 2/5     & 8/14  \\
  \hline
\end{tabular}

\begin{equation*}
  \begin{split}
    P(Play=Yes) & = 9/14 \\
    P(Play=No)  & = 5/14 \\
  \end{split}
\end{equation*}

Let x' denote the conditions of playing tennis or not

$x' = (Outlook=Rain, Temperature=Mild, Humidity=Normal, Wind=Weak)$

The probability that tennis is played lookup table

\begin{equation*}
  \begin{split}
    P(Outlook=Rain|Play=Yes)     & = 3/9 \\
    P(Temperature=Mild|Play=Yes) & = 4/9 \\
    P(Humidity=Normal|Play=Yes)  & = 6/9 \\
    P(Wind=Weak|Play=Yes)        & = 6/9 \\
  \end{split}
\end{equation*}

\begin{equation*}
  \begin{split}
    P(Outlook=Rain|Play=No)     & = 2/5 \\
    P(Temperature=Mild|Play=No) & = 2/5 \\
    P(Humidity=Normal|Play=No)  & = 1/5 \\
    P(Wind=Weak|Play=No)        & = 2/5 \\
  \end{split}
\end{equation*}

Now we can construnct the following equation

\begin{equation*}
  \begin{split}
    P(Play=Yes|x') & = P(x'|Play=Yes)P(Play=Yes) \\
    & = [P(Rain|Yes)P(Mild|Yes)P(Normal|Yes)P(Weak|Yes)]P(Play=Yes) \\
    & = (3/9 * 4/9 * 6/9 * 6/9)(9/14) \\
    & = 0.0423280423 \\
  \end{split}
\end{equation*}

\begin{equation*}
  \begin{split}
    P(Play=No|x')  & = P(x'|Play=No)P(Play=No) \\
    & = [P(Rain|No)P(Mild|No)P(Normal|No)P(Weak|No)]P(Play=No) \\
    & = (2/5 * 2/5 * 1/5 * 2/5)(5/14) \\
    & = 0.0045714286 \\
  \end{split}
\end{equation*}

Given The Fact that $P(Play=Yes|x') > P(Play=No|x')$ we would label x' to be Yes

\section{Part 2}

	The	humidity	value	could	be	continuous	practically.	In	the	above	data	set,	if
the	humidity	value	is	as	follows per	original	data	set	order:
\begin{equation*}
  \begin{split}
    Yes: & 65.7,	20.7,	5.1,	6.9,	4.8,	6.9,	8.7,	10.4,	15.3, \\
    No:	& 58.1,	66.4,	6.5,	10.5,	12.8 \\
  \end{split}
\end{equation*}
Please	use	Naïve	Bayes	to	help	make	decision	on	playing	tennis	or	not	when	the
Outlook	is	Overcast,	Temperature	is	mild,	Humidity	is	8.8,	and	Wind	is	weak.

\chapter{Question 3}

We  have  a training  dataset as  follows:

\vspace*{0.5 cm}

\begin{tabular}{ |c|c|c| }
  \hline
  Feature 1 & Feature 2 & Label \\
  \hline
  6         & 6         & L1    \\
  \hline
  6         & 4         & L1    \\
  \hline
  2         & 3         & L2    \\
  \hline
  1         & 4         & L2    \\
  \hline
\end{tabular}

\vspace*{0.5 cm}

Using K-NN  algorithm to  determine the label for a new data  record  (3, 6). The 
similarity  measure is  assumed to  be  Euclidean distance.

\vspace*{0.5 cm}

\begin{tikzpicture}
  \begin{axis}[%
      scatter/classes={%
        L1={color=blue},
        L2={color=red},
        L3={color=black}
    }]
    \addplot[scatter,only marks,%
    scatter src=explicit symbolic]%
    table[meta=label] {
      x y label
      6 6 L1
      6 4 L1
      2 3 L2
      1 4 L2
      3 6 L3
    };
  \end{axis}
\end{tikzpicture}

Using Euclidean Distance we can calculate the distances between the two points

\begin{equation*}
  d(x_i, x_j) = \sqrt{\sum_{r=1}^{n}(f_r(x_i)-f_r(x_j))^2} 
\end{equation*}

Using Euclidean Distance we get the following results

\begin{equation*}
  \begin{split}
    (6, 6, L1): \sqrt{(6-3)^2 + (6-6)^2} & = \sqrt{9} = 3 \\
    (6, 4, L1): \sqrt{(6-3)^2 + (4-6)^2} & = \sqrt{13} = 3.605551275 \\
    (2, 3, L2): \sqrt{(2-3)^2 + (3-6)^2} & = \sqrt{10} = 3.16227766 \\
    (1, 4, L2): \sqrt{(1-3)^2 + (4-6)^2} & = \sqrt{8} = 2.828427125 \\
  \end{split}
\end{equation*}

Using K=1 we can see that the new data record of (3, 6) would be classified as L2

Using K=2 the new data record is split between L1 and L2

Using K=3 we can see that the new data record of (3, 6) would be classified as L2

Using K=4 the new data record is split between L1 and L2

\chapter{Question 4}

Given the following 12  data  points x=(x1, x2) in  the training  data  set.

\begin{equation*}
  (1, 3), (2, 5), (3, 15),  (4, 8), (4, 19),  (5, 6), (6, 13), \\
  (8, 6), (10,  15),  (12,  7), (15,  4), (17,  18)
\end{equation*}

\section{Part 1}
If  we  know  the value range of  features  x1, x2  is  in  [0, 20],  please  build the 
quadtree. 

Quadtree algorithm, A quadtree for a set of points P in a square
\begin{equation*}
  \begin{split}
    Q & = [x_{1q} : x_{2q}] * [y_{1q} : y_{2q}] \\
  \end{split}
\end{equation*}

\begin{equation*}
  \begin{split}
    Q_{NE} & = \{p \in P : p_x > x_{mid} \land p_y > y_{mid}\} \\
    Q_{NW} & = \{p \in P : p_x \leq x_{mid} \land p_y > y_{mid}\} \\
    Q_{SW} & = \{p \in P : p_x \leq x_{mid} \land p_y \leq y_{mid}\} \\
    Q_{SE} & = \{p \in P : p_x > x_{mid} \land p_y \leq y_{mid}\} \\
  \end{split}
\end{equation*}

\begin{equation*}
  \begin{split}
    x_{mid} & = (x_{1q} + x_{2q})/2 \\
    y_{mid} & = (y_{1q} + y_{2q})/2 \\
  \end{split}
\end{equation*}

The starting root quadrents are

\begin{equation*}
  \begin{split}
    x_{mid} & = 10 \\
    y_{mid} & = 10
  \end{split}
\end{equation*}

Given the points we can calculate what quadrent they are in

  \pgfkeys{
    /prepare label/.style={
      /print label/\detokenize{#1}/.code={\ttfamily\detokenize{#1}}
    },
    /prepare label/.list={a,b,c,d,e,f,g,h,i,j,k,l}
  }

  \begin{tikzpicture}
    \begin{axis}[
          xmax=20,
          xmin=0,
          ymax=20,
          ymin=0,
      ]
      \addplot[
        mark=*,
        only marks,
        point meta=explicit symbolic,
        nodes near coords={
          \pgfkeys{/print label/\pgfplotspointmeta/.try}
        }
      ]
      table[header=false,meta index=0,x index=1,y index=2]{
        a   1  3
        b   2  5
        c   3 15
        d   4  8
        e   4 19
        f   5  6
        g   6 13
        h   8  6
        i  10 15
        j  12  7
        k  15  4
        l  17 18
      };
    \end{axis}
  \end{tikzpicture}

  \break

  \begin{multicols}{2}
    \begin{equation*}
      \begin{split}
        a : (1, 3) : & \\
             Q_{NE} & = \{1 > 10 \land 3 > 10\} \\
             Q_{NW} & = \{1 \leq 10 \land 3 > 10\} \\
             -> Q_{SW} & = \{1 \leq 10 \land 3 \leq 10\} \\
             Q_{SE} & = \{1 > 10 \land 3 \leq 10\} \\
        b : (2, 5) : & \\
             Q_{NE} & = \{2 > 10 \land 5 > 10\} \\
             Q_{NW} & = \{2 \leq 10 \land 5 > 10\} \\
             ->Q_{SW} & = \{2 \leq 10 \land 5 \leq 10\} \\
             Q_{SE} & = \{2 > 10 \land 5 \leq 10\} \\
        c : (3, 15) : \\
             Q_{NE} & = \{3 > 10 \land 15 > 10\} \\
             ->Q_{NW} & = \{3 \leq 10 \land 15 > 10\} \\
             Q_{SW} & = \{3 \leq 10 \land 15 \leq 10\} \\
             Q_{SE} & = \{3 > 10 \land 15 \leq 10\} \\
        d : (4, 8) : \\
             Q_{NE} & = \{4 > 10 \land 8 > 10\} \\
             Q_{NW} & = \{4 \leq 10 \land 8 > 10\} \\
             ->Q_{SW} & = \{4 \leq 10 \land 8 \leq 10\} \\
             Q_{SE} & = \{4 > 10 \land 8 \leq 10\} \\
        e : (4, 19) : \\
             Q_{NE} & = \{4 > 10 \land 19 > 10\} \\
             ->Q_{NW} & = \{4 \leq 10 \land 19 > 10\} \\
             Q_{SW} & = \{4 \leq 10 \land 19 \leq 10\} \\
             Q_{SE} & = \{4 > 10 \land 19 \leq 10\} \\
        f : (5, 6) : \\
             Q_{NE} & = \{5 > 10 \land 6 > 10\} \\
             Q_{NW} & = \{5 \leq 10 \land 6 > 10\} \\
             ->Q_{SW} & = \{5 \leq 10 \land 6 \leq 10\} \\
             Q_{SE} & = \{5 > 10 \land 6 \leq 10\} \\
      \end{split}
    \end{equation*}\break
      \begin{equation*}
        \begin{split}
          g : (6, 13) : \\
             Q_{NE} & = \{6 > 10 \land 13 > 10\} \\
             ->Q_{NW} & = \{6 \leq 10 \land 13 > 10\} \\
             Q_{SW} & = \{6 \leq 10 \land 13 \leq 10\} \\
             Q_{SE} & = \{6 > 10 \land 13 \leq 10\} \\
          h : (8, 6) : \\
             Q_{NE} & = \{8 > 10 \land 6 > 10\} \\
             Q_{NW} & = \{8 \leq 10 \land 6 > 10\} \\
             ->Q_{SW} & = \{8 \leq 10 \land 6 \leq 10\} \\
             Q_{SE} & = \{8 > 10 \land 6 \leq 10\} \\
          i : (10, 15) : \\
             Q_{NE} & = \{10 > 10 \land 15 > 10\} \\
             ->Q_{NW} & = \{10 \leq 10 \land 15 > 10\} \\
             Q_{SW} & = \{10 \leq 10 \land 15 \leq 10\} \\
             Q_{SE} & = \{10 > 10 \land 15 \leq 10\} \\
          j : (12, 7) : \\
             Q_{NE} & = \{12 > 10 \land 7 > 10\} \\
             Q_{NW} & = \{12 \leq 10 \land 7 > 10\} \\
             Q_{SW} & = \{12 \leq 10 \land 7 \leq 10\} \\
             ->Q_{SE} & = \{12 > 10 \land 7 \leq 10\} \\
          k : (15, 4) : \\
             Q_{NE} & = \{15 > 10 \land 4 > 10\} \\
             Q_{NW} & = \{15 \leq 10 \land 4 > 10\} \\
             Q_{SW} & = \{15 \leq 10 \land 4 \leq 10\} \\
             ->Q_{SE} & = \{15 > 10 \land 4 \leq 10\} \\
          l : (17, 18) : \\
             ->Q_{NE} & = \{17 > 10 \land 18 > 10\} \\
             Q_{NW} & = \{17 \leq 10 \land 18 > 10\} \\
             Q_{SW} & = \{17 \leq 10 \land 18 \leq 10\} \\
             Q_{SE} & = \{17 > 10 \land 18 \leq 10\} \\
        \end{split}
      \end{equation*}
  \end{multicols}

  \break

  Now we can generate are first graph using the root split

  \begin{tikzpicture}
    \begin{axis}[
          xmax=20,
          xmin=0,
          ymax=20,
          ymin=0,
      ]
      \addplot[
        mark=*,
        only marks,
        point meta=explicit symbolic,
        nodes near coords={
          \pgfkeys{/print label/\pgfplotspointmeta/.try}
        }
      ]
      table[header=false,meta index=0,x index=1,y index=2]{
        a   1  3
        b   2  5
        c   3 15
        d   4  8
        e   4 19
        f   5  6
        g   6 13
        h   8  6
        i  10 15
        j  12  7
        k  15  4
        l  17 18
      };
      \draw [very thick,dotted] (100,0) -- (100,200);
      \draw [very thick,dotted] (0,100) -- (200,100);
    \end{axis}
  \end{tikzpicture}

\pgfmathsetmacro{\wdbox}{.02\textwidth}
\tikzset{
  >= stealth, 
  every picture/.style={ultra thick},
  every node/.style={anchor=north},
  simple/.style={draw,minimum size=3*\wdbox,scale=.8},
  array/.style={%
              draw,scale=.8, 
              inner sep=\wdbox, 
              rounded corners,
              rectangle, 
              rectangle split, 
              rectangle split parts=4,
              rectangle split ignore empty parts=false, 
              rectangle split horizontal,
              append after command={%
              \pgfextra{\let\mainnode=\tikzlastnode} 
              coordinate (c1 \mainnode) at ($(\mainnode.west)!.5!(\mainnode.one split)$)
              coordinate (c2 \mainnode) at ($(\mainnode.one split)!.5!(\mainnode.two split)$)
              coordinate (c3 \mainnode) at ($(\mainnode.two split)!.5!(\mainnode.three split)$)
              coordinate (c4 \mainnode) at ($(\mainnode.three split)!.5!(\mainnode.east)$)                
                    }
                  }
                }
  \begin{tikzpicture}[x=.035\textwidth,y=.035\textwidth]
       \node[simple] (1) {X 10, Y 10};
       \draw[->] (1.center) ++(0, -10pt) -- +(-4,-2) node[simple] (2) {a,b,d,f,h};
       \draw[->] (1.center) ++(0, -10pt) -- +(-1,-2) node[simple] (3) {j,k};  
       \draw[->] (1.center) ++(0, -10pt) -- +(+1,-2) node[simple] (4) {l};
       \draw[->] (1.center) ++(0, -10pt) -- +(+4,-2) node[simple] (5) {e,c,g,i};
  \end{tikzpicture}

Now that we can see that $Q_{SW}$ and $Q_{NW}$ has to many elements so we will break down these two quadrants. First we need to calculate the new mid points for the two quadrant.

\begin{equation*}
  \begin{split}
  Q_{SW} : & \\
    x_{mid} & = 5 \\
    y_{mid} & = 5 \\
   Q_{NW} : & \\
    x_{mid} & = 5 \\
    y_{mid} & = 15
  \end{split}
\end{equation*}

Given the points we can calculate what points are in the new quadrants

  \pgfkeys{
    /prepare label/.style={
      /print label/\detokenize{#1}/.code={\ttfamily\detokenize{#1}}
    },
    /prepare label/.list={a,b,c,d,e,f,g,h,i,j,k,l,n}
  }

  \begin{tikzpicture}
    \begin{axis}[
          xmax=20,
          xmin=0,
          ymax=20,
          ymin=0,
      ]
      \addplot[
        mark=*,
        only marks,
        point meta=explicit symbolic,
        nodes near coords={
          \pgfkeys{/print label/\pgfplotspointmeta/.try}
        }
      ]
      table[header=false,meta index=0,x index=1,y index=2]{
        a   1  3
        b   2  5
        c   3 15
        d   4  8
        e   4 19
        f   5  6
        g   6 13
        h   8  6
        i  10 15
        j  12  7
        k  15  4
        l  17 18
      };
       \draw [very thick,dotted] (100,0) -- (100,200);
      \draw [very thick,dotted] (0,100) -- (200,100);
      
       \draw [very thick,dotted] (50,0) -- (50,100);
      \draw [very thick,dotted] (0,50) -- (100,50);
      
      \draw [very thick,dotted] (50,100) -- (50,200);
      \draw [very thick,dotted] (0,150) -- (100,150);
    \end{axis}
  \end{tikzpicture}

  \break

  \begin{multicols}{2}
    \begin{equation*}
      \begin{split}
      Q_{SW} : (5, 5) & \\
        a : (1, 3) : & \\
             Q_{NE} & = \{1 > 5 \land 3 > 5\} \\
             Q_{NW} & = \{1 \leq 5 \land 3 > 5\} \\
             ->Q_{SW} & = \{1 \leq 5 \land 3 \leq 5\} \\
             Q_{SE} & = \{1 > 5 \land 3 \leq 5\} \\
        b : (2, 5) : & \\
             Q_{NE} & = \{2 > 5 \land 5 > 5\} \\
             Q_{NW} & = \{2 \leq 5 \land 5 > 5\} \\
             ->Q_{SW} & = \{2 \leq 5 \land 5 \leq 5\} \\
             Q_{SE} & = \{2 > 5 \land 5 \leq 5\} \\
        d : (4, 8) : \\
             Q_{NE} & = \{4 > 5 \land 8 > 5\} \\
             ->Q_{NW} & = \{4 \leq 5 \land 8 > 5\} \\
             Q_{SW} & = \{4 \leq 5 \land 8 \leq 5\} \\
             Q_{SE} & = \{4 > 5 \land 8 \leq 5\} \\
        f : (5, 6) : \\
             Q_{NE} & = \{5 > 5 \land 6 > 5\} \\
             ->Q_{NW} & = \{5 \leq 5 \land 6 > 5\} \\
             Q_{SW} & = \{5 \leq 5 \land 6 \leq 5\} \\
             Q_{SE} & = \{5 > 5 \land 6 \leq 5\} \\
        h : (8, 6) : \\
             ->Q_{NE} & = \{8 > 5 \land 6 > 5\} \\
             Q_{NW} & = \{8 \leq 5 \land 6 > 5\} \\
             Q_{SW} & = \{8 \leq 5 \land 6 \leq 5\} \\
             Q_{SE} & = \{8 > 5 \land 6 \leq 5\} \\
      \end{split}
    \end{equation*}\break
      \begin{equation*}
        \begin{split}
        Q_{NW} : (5, 15) & \\
        c : (3, 15) : \\
             Q_{NE} & = \{3 > 5 \land 15 > 15\} \\
             Q_{NW} & = \{3 \leq 5 \land 15 > 15\} \\
             ->Q_{SW} & = \{3 \leq 5 \land 15 \leq 15\} \\
             Q_{SE} & = \{3 > 5 \land 15 \leq 15\} \\
        e : (4, 19) : \\
             Q_{NE} & = \{4 > 5 \land 19 > 15\} \\
             ->Q_{NW} & = \{4 \leq 5 \land 19 > 15\} \\
             Q_{SW} & = \{4 \leq 5 \land 19 \leq 15\} \\
             Q_{SE} & = \{4 > 5 \land 19 \leq 15\} \\
        g : (6, 13) : \\
             Q_{NE} & = \{6 > 5 \land 13 > 15\} \\
             Q_{NW} & = \{6 \leq 5 \land 13 > 15\} \\
             Q_{SW} & = \{6 \leq 5 \land 13 \leq 15\} \\
             ->Q_{SE} & = \{6 > 5 \land 13 \leq 15\} \\
          i : (10, 15) : \\
             Q_{NE} & = \{10 > 5 \land 15 > 15\} \\
             Q_{NW} & = \{10 \leq 5 \land 15 > 15\} \\
             Q_{SW} & = \{10 \leq 5 \land 15 \leq 15\} \\
             ->Q_{SE} & = \{10 > 5 \land 15 \leq 15\} \\
        \end{split}
      \end{equation*}
  \end{multicols}
  
  The final quadtree looks like the following.
  
  \vspace{1cm}
  
\begin{tikzpicture}[x=.035\textwidth,y=.035\textwidth]
       \node[simple] (1) {X 10, Y 10};
       \draw[->] (1.center) ++(0, -10pt) -- +(-4,-2) node[simple] (2) {X 5, Y 5};
       
       \draw[->] (2.center) ++(0, -10pt) -- +(-3,-2) node[simple] (3) {a,b}; 
       \draw[->] (2.center) ++(0, -10pt) -- +(-1,-2) node[simple] (3) {$\emptyset$}; 
       \draw[->] (2.center) ++(0, -10pt) -- +(+1,-2) node[simple] (3) {h}; 
       \draw[->] (2.center) ++(0, -10pt) -- +(+3,-2) node[simple] (3) {d,f}; 
       
       \draw[->] (1.center) ++(0, -10pt) -- +(-1,-2) node[simple] (3) {j,k}; 
       \draw[->] (1.center) ++(0, -10pt) -- +(+1,-2) node[simple] (4) {l};
       \draw[->] (1.center) ++(0, -10pt) -- +(+4,-2) node[simple] (5) {X 5, Y 15};
       
       \draw[->] (5.center) ++(0, -10pt) -- +(-3,-2) node[simple] (3) {c}; 
       \draw[->] (5.center) ++(0, -10pt) -- +(-1,-2) node[simple] (3) {g,i}; 
       \draw[->] (5.center) ++(0, -10pt) -- +(+1,-2) node[simple] (3) {$\emptyset$}; 
       \draw[->] (5.center) ++(0, -10pt) -- +(+3,-2) node[simple] (3) {e}; 
  \end{tikzpicture}

\section{Part 2}
Using the quadtree  learnt  in  4.1 to  find  the nearest neighbor  of  data  point 
(11,  16).

\begin{tikzpicture}
    \begin{axis}[
          xmax=20,
          xmin=0,
          ymax=20,
          ymin=0,
      ]
      \addplot[
        mark=*,
        only marks,
        point meta=explicit symbolic,
        nodes near coords={
          \pgfkeys{/print label/\pgfplotspointmeta/.try}
        }
      ]
      table[header=false,meta index=0,x index=1,y index=2]{
        a   1  3
        b   2  5
        c   3 15
        d   4  8
        e   4 19
        f   5  6
        g   6 13
        h   8  6
        i  10 15
        j  12  7
        k  15  4
        l  17 18
        n 11 16
      };
       \draw [very thick,dotted] (100,0) -- (100,200);
      \draw [very thick,dotted] (0,100) -- (200,100);
      
       \draw [very thick,dotted] (50,0) -- (50,100);
      \draw [very thick,dotted] (0,50) -- (100,50);
      
      \draw [very thick,dotted] (50,100) -- (50,200);
      \draw [very thick,dotted] (0,150) -- (100,150);
    \end{axis}
  \end{tikzpicture}

Given the tree from problem 4.1 you can see that are new point n would be in the NE quadrant with l which would make l its closest neighbor based on the graph but if you look at the Euclidian distance you can see that i is actually the closet neighbor to n 

\chapter{Question 5}

Given the following 12 data points x=(x1,	x2)	in	the	training	data	set.
\begin{equation*}
  (1,	3),	(2,	5),	(3,	15),	(4,	8),	(4,	19),	(5,	6),	(6,	13), \\
(8,	6),	(10,	15),	(12,	7),	(15,	4),	(17,	18).
\end{equation*}

\section{Part 1}
please	build	the	kd-tree.

\begin{tikzpicture}
    \begin{axis}[
          xmax=20,
          xmin=0,
          ymax=20,
          ymin=0,
      ]
      \addplot[
        mark=*,
        only marks,
        point meta=explicit symbolic,
        nodes near coords={
          \pgfkeys{/print label/\pgfplotspointmeta/.try}
        }
      ]
      table[header=false,meta index=0,x index=1,y index=2]{
        a   1  3
        b   2  5
        c   3 15
        d   4  8
        e   4 19
        f   5  6
        g   6 13
        h   8  6
        i  10 15
        j  12  7
        k  15  4
        l  17 18
      };
    \end{axis}
  \end{tikzpicture}

  First we will construct the boundry lines then construct the graph from that.
Starting with x we will compute are first boundary the we will compute the y median and repeat. 

\begin{equation*}
  x_{med} = \frac{1+2+3+4+4+5+6+8+10+12+15+17}{12}=\frac{87}{12} = 7.25
\end{equation*}

\begin{tikzpicture}
    \begin{axis}[
          xmax=20,
          xmin=0,
          ymax=20,
          ymin=0,
      ]
      \addplot[
        mark=*,
        only marks,
        point meta=explicit symbolic,
        nodes near coords={
          \pgfkeys{/print label/\pgfplotspointmeta/.try}
        }
      ]
      table[header=false,meta index=0,x index=1,y index=2]{
        a   1  3
        b   2  5
        c   3 15
        d   4  8
        e   4 19
        f   5  6
        g   6 13
        h   8  6
        i  10 15
        j  12  7
        k  15  4
        l  17 18
      };
      \draw [very thick,dotted] (72.5,0) -- (72.5,200);
    \end{axis}
  \end{tikzpicture}

\begin{equation*}
  y1_{med} = \frac{3+5+15+8+19+6+13}{7}=\frac{69}{7} = 9.8571
  y2_{med} = \frac{6+15+7+4+18}{5}=\frac{50}{5} = 10
\end{equation*}

\begin{tikzpicture}
    \begin{axis}[
          xmax=20,
          xmin=0,
          ymax=20,
          ymin=0,
      ]
      \addplot[
        mark=*,
        only marks,
        point meta=explicit symbolic,
        nodes near coords={
          \pgfkeys{/print label/\pgfplotspointmeta/.try}
        }
      ]
      table[header=false,meta index=0,x index=1,y index=2]{
        a   1  3
        b   2  5
        c   3 15
        d   4  8
        e   4 19
        f   5  6
        g   6 13
        h   8  6
        i  10 15
        j  12  7
        k  15  4
        l  17 18
      };
      \draw [very thick,dotted] (72.5,0) -- (72.5,200);
      \draw [very thick,dotted] (0,98.571) -- (72.5,98.571);
      \draw [very thick,dotted] (72.5,100) -- (200,100);
    \end{axis}
  \end{tikzpicture}

  \begin{equation*}
    \begin{split}
      x1_{med} & = \frac{1+2+4+5}{4}=\frac{12}{4} = 3 \\
      x2_{med} & = \frac{8+12+15}{3}=\frac{35}{3} = 11.667 \\
      x3_{med} & = \frac{10+17}{2}=\frac{27}{2} = 13.5 \\
      x4_{med} & = \frac{3+4+6}{3}=\frac{13}{3} = 4.333 \\
    \end{split}
  \end{equation*}

\begin{tikzpicture}
    \begin{axis}[
          xmax=20,
          xmin=0,
          ymax=20,
          ymin=0,
      ]
      \addplot[
        mark=*,
        only marks,
        point meta=explicit symbolic,
        nodes near coords={
          \pgfkeys{/print label/\pgfplotspointmeta/.try}
        }
      ]
      table[header=false,meta index=0,x index=1,y index=2]{
        a   1  3
        b   2  5
        c   3 15
        d   4  8
        e   4 19
        f   5  6
        g   6 13
        h   8  6
        i  10 15
        j  12  7
        k  15  4
        l  17 18
      };
      \draw [very thick,dotted] (72.5,0) -- (72.5,200);

      \draw [very thick,dotted] (0,98.571) -- (72.5,98.571);
      \draw [very thick,dotted] (72.5,100) -- (200,100);

      \draw [very thick,dotted] (30,0) -- (30,99.1667);
      \draw [very thick,dotted] (116.67,0) -- (116.67,99.1667);
      \draw [very thick,dotted] (135,99.1667) -- (135,200);
      \draw [very thick,dotted] (43.33,99.1667) -- (43.33,200);

    \end{axis}
  \end{tikzpicture}

\begin{equation*}
  y1_{med} = \frac{15+19}{2}=\frac{34}{2} = 17
  y2_{med} = \frac{3+5}{2}=\frac{8}{2} = 4
  y3_{med} = \frac{8+6}{2}=\frac{14}{2} = 7
  y4_{med} = \frac{7+4}{2}=\frac{11}{2} = 5.5
\end{equation*}

\begin{tikzpicture}
    \begin{axis}[
          xmax=20,
          xmin=0,
          ymax=20,
          ymin=0,
      ]
      \addplot[
        mark=*,
        only marks,
        point meta=explicit symbolic,
        nodes near coords={
          \pgfkeys{/print label/\pgfplotspointmeta/.try}
        }
      ]
      table[header=false,meta index=0,x index=1,y index=2]{
        a   1  3
        b   2  5
        c   3 15
        d   4  8
        e   4 19
        f   5  6
        g   6 13
        h   8  6
        i  10 15
        j  12  7
        k  15  4
        l  17 18
      };
      \draw [very thick,dotted] (72.5,0) -- (72.5,200);

      \draw [very thick,dotted] (0,98.571) -- (72.5,98.571);
      \draw [very thick,dotted] (72.5,100) -- (200,100);

      \draw [very thick,dotted] (30,0) -- (30,99.1667);
      \draw [very thick,dotted] (116.67,0) -- (116.67,99.1667);
      \draw [very thick,dotted] (135,99.1667) -- (135,200);
      \draw [very thick,dotted] (43.33,99.1667) -- (43.33,200);

      \draw [very thick,dotted] (0,170) -- (43.33,170);
      \draw [very thick,dotted] (0,40) -- (30,40);
      \draw [very thick,dotted] (30,70) -- (72.5,70);
      \draw [very thick,dotted] (116.67,55) -- (200,55);
    \end{axis}
  \end{tikzpicture}

  With all this infomation we can construct the graph

\begin{tikzpicture}[x=.035\textwidth,y=.035\textwidth]
       \node[simple] (1) {X 10};
       
      \draw[->] (1.center) ++(0, -10pt) -- +(-6,-2) node[simple] (2) {Y 9.8571}; 
       \draw[->] (1.center) ++(0, -10pt) -- +(+6,-2) node[simple] (3) {Y 10}; 

       \draw[->] (2.center) ++(0, -10pt) -- +(-4,-2) node[simple] (4) {X 3}; 
       \draw[->] (2.center) ++(0, -10pt) -- +(+4,-2) node[simple] (5) {X 4.33}; 
       \draw[->] (3.center) ++(0, -10pt) -- +(-4,-2) node[simple] (6) {X 11.667}; 
       \draw[->] (3.center) ++(0, -10pt) -- +(+4,-2) node[simple] (7) {X 13.5}; 

       \draw[->] (4.center) ++(0, -10pt) -- +(-2,-2) node[simple] (8) {Y 4}; 
       \draw[->] (4.center) ++(0, -10pt) -- +(+2,-2) node[simple] (9) {Y 7}; 
       \draw[->] (5.center) ++(0, -10pt) -- +(-2,-2) node[simple] (10) {Y 17}; 
       \draw[->] (6.center) ++(0, -10pt) -- +(+2,-2) node[simple] (11) {Y 5.5}; 

       \draw[->] (5.center) ++(0, -10pt) -- +(0,-2) node[simple] (18) {g}; 
       \draw[->] (6.center) ++(0, -10pt) -- +(0,-2) node[simple] (19) {h}; 
       \draw[->] (7.center) ++(0, -10pt) -- +(-1,-2) node[simple] (19) {i}; 
       \draw[->] (7.center) ++(0, -10pt) -- +(+1,-2) node[simple] (19) {l}; 
       \draw[->] (8.center) ++(0, -10pt) -- +(-1,-2) node[simple] (12) {a}; 
       \draw[->] (8.center) ++(0, -10pt) -- +(+1,-2) node[simple] (13) {b}; 
       \draw[->] (9.center) ++(0, -10pt) -- +(-1,-2) node[simple] (17) {f}; 
       \draw[->] (9.center) ++(0, -10pt) -- +(+1,-2) node[simple] (15) {d}; 
       \draw[->] (10.center) ++(0, -10pt) -- +(-1,-2) node[simple] (14) {c}; 
       \draw[->] (10.center) ++(0, -10pt) -- +(+1,-2) node[simple] (16) {e}; 
       \draw[->] (11.center) ++(0, -10pt) -- +(-1,-2) node[simple] (19) {k}; 
       \draw[->] (11.center) ++(0, -10pt) -- +(+1,-2) node[simple] (19) {j}; 
       
  \end{tikzpicture}

\section{Part 2}	
using the	kd-tree	learnt	in	5.1	to	find	the	nearest	neighbor	of	data	point	(11,	
16).

\begin{tikzpicture}
    \begin{axis}[
          xmax=20,
          xmin=0,
          ymax=20,
          ymin=0,
      ]
      \addplot[
        mark=*,
        only marks,
        point meta=explicit symbolic,
        nodes near coords={
          \pgfkeys{/print label/\pgfplotspointmeta/.try}
        }
      ]
      table[header=false,meta index=0,x index=1,y index=2]{
        a   1  3
        b   2  5
        c   3 15
        d   4  8
        e   4 19
        f   5  6
        g   6 13
        h   8  6
        i  10 15
        j  12  7
        k  15  4
        l  17 18
        n  11 16
      };
      \draw [very thick,dotted] (72.5,0) -- (72.5,200);

      \draw [very thick,dotted] (0,98.571) -- (72.5,98.571);
      \draw [very thick,dotted] (72.5,100) -- (200,100);

      \draw [very thick,dotted] (30,0) -- (30,99.1667);
      \draw [very thick,dotted] (116.67,0) -- (116.67,99.1667);
      \draw [very thick,dotted] (135,99.1667) -- (135,200);
      \draw [very thick,dotted] (43.33,99.1667) -- (43.33,200);

      \draw [very thick,dotted] (0,170) -- (43.33,170);
      \draw [very thick,dotted] (0,40) -- (30,40);
      \draw [very thick,dotted] (30,70) -- (72.5,70);
      \draw [very thick,dotted] (116.67,55) -- (200,55);
    \end{axis}
  \end{tikzpicture}

  Using the kd-tree to find nearest neighbor we will traverse the tree and end up on the far right side on the leaf node l. Checking the surounding area we can see that l is the true nearest neighbor.


\chapter{Question 6}

For	the	play	tennis	data	set	shown	below.

\begin{tabular}{ |c||c c c c | c | }
  \hline
  Day & Outlook & Temperature & Humidity & Wind & PlayTennis \\
  \hline
  D1  & Sunny                     & Hot                   & High                    & Weak                    & No                   \\
  D2  & Sunny                     & Hot                   & High                    & Strong                  & No                   \\
  D3  & \textcolor{red}{Overcast} & \textcolor{red}{Hot}  & \textcolor{red}{High}   & \textcolor{red}{Weak}   & \textcolor{red}{Yes} \\
  D4  & \textcolor{red}{Rain}     & \textcolor{red}{Mild} & \textcolor{red}{High}   & \textcolor{red}{Weak}   & \textcolor{red}{Yes} \\
  D5  & \textcolor{red}{Rain}     & \textcolor{red}{Cool} & \textcolor{red}{Normal} & \textcolor{red}{Weak}   & \textcolor{red}{Yes} \\
  D6  & Rain                      & Cool                  & Normal                  & Strong                  & No                   \\
  D7  & \textcolor{red}{Overcast} & \textcolor{red}{Cool} & \textcolor{red}{Normal} & \textcolor{red}{Strong} & \textcolor{red}{Yes} \\
  D8  & Sunny                     & Mild                  & High                    & Weak                    & No                   \\
  D9  & \textcolor{red}{Sunny}    & \textcolor{red}{Cool} & \textcolor{red}{Normal} & \textcolor{red}{Weak}   & \textcolor{red}{Yes} \\
  D10 & \textcolor{red}{Rain}     & \textcolor{red}{Mild} & \textcolor{red}{Normal} & \textcolor{red}{Weak}   & \textcolor{red}{Yes} \\
  D11 & \textcolor{red}{Sunny}    & \textcolor{red}{Mild} & \textcolor{red}{Normal} & \textcolor{red}{Strong} & \textcolor{red}{Yes} \\
  D12 & \textcolor{red}{Overcast} & \textcolor{red}{Mild} & \textcolor{red}{High}   & \textcolor{red}{Strong} & \textcolor{red}{Yes} \\
  D13 & \textcolor{red}{Overcast} & \textcolor{red}{Hot}  & \textcolor{red}{Normal} & \textcolor{red}{Weak}   & \textcolor{red}{Yes} \\
  D14 & Rain                      & Mild                  & High                    & Strong                  & No                   \\
  \hline
\end{tabular}

\section{Part 1}
build	the	decision	tree	by	using	information	gain.

First we must calculate the entropy of all the classes

\begin{tabular}{ |c|c|c|c| }
  \hline
  \textbf{OUTLOOK} & Play=Yes & Play=No & Total \\
  \hline
  Sunny            & 2/9      & 3/5     & 5/14  \\
  \hline
  Overcast         & 4/9      & 0/5     & 4/14  \\
  \hline
  Rain             & 3/9      & 2/5     & 5/14  \\
  \hline
\end{tabular}

\vspace*{1 cm}

\begin{tabular}{ |c|c|c|c| }
  \hline
  \textbf{TEMPERATURE} & Play=Yes & Play=No & Total \\
  \hline
  Hot                  & 2/9      & 2/5     & 4/14  \\
  \hline
  Mild                 & 4/9      & 2/5     & 6/14  \\
  \hline
  Cool                 & 3/9      & 1/5     & 4/14  \\
  \hline
\end{tabular}

\vspace*{1 cm}

\begin{tabular}{ |c|c|c|c| }
  \hline
  \textbf{HUMIDITY} & Play=Yes & Play=No & Total \\
  \hline
  High              & 3/9      & 4/5     & 7/14  \\
  \hline
  Normal            & 6/9      & 1/5     & 7/14  \\
  \hline
\end{tabular}

\vspace*{1 cm}

\begin{tabular}{ |c|c|c|c| }
  \hline
  \textbf{WIND} & Play=Yes & Play=No & Total \\
  \hline
  Strong        & 3/9      & 3/5     & 6/14  \\
  \hline
  Weak          & 6/9      & 2/5     & 8/14  \\
  \hline
\end{tabular}

\begin{equation*}
  \begin{split}
    P(Play=Yes) & = 9/14 \\
    P(Play=No)  & = 5/14 \\
  \end{split}
\end{equation*}

\begin{equation*}
  \begin{split}
    Entropy(PlayTennis) & = -(5/14 * \log_2 5/14) - (9/14 * \log_2 9/14) = 0.94 \\
    Entropy(PlayTennis, Outlook) & \\
    & = P(Sunny) * Entropy(Sunny) + P(Overcast) * Entropy(Overcast) + P(Rainy) * Entropy(Rainy)  \\
    & = (5/14) * 0.971 + (4/14) * 0.0 + (5/14)*0.971 = 0.693 \\
    Entropy(PlayTennis, Temperature) &  \\
    & = P(Hot) * Entropy(Hot) + P(Mild) * Entropy(Mild) + P(Cool) * Entropy(Cool) \\
    & = (4/14) * 1 + (6/14) * 0.9183 + (4/14)*0.811 = 0.911 \\
    Entropy(PlayTennis, Humidity) &  \\
    & = P(High) * Entropy(High) + P(Normal) * Entropy(Normal) \\
    & = (7/14) * 0.985 + (7/14) * 0.5917 = 0.78835 \\
    Entropy(PlayTennis, Wind) &  \\
    & = P(Strong) * Entropy(Srong) + P(Weak) * Entropy(Weak) \\
    & = (6/14) * 1 + (8/14) * 0.8113 = 0.8922 \\
  \end{split}
\end{equation*}

\begin{equation*}
  \begin{split}
    Gain(PlayTennis, Outlook) & \\ 
    & = Entropy(PlayTennis) - Entropy(PlayTennis, Outlook) \\
    & = 0.94 - 0.693 = 0.247 \\
    Gain(PlayTennis, Temperature) & \\
    & = Entropy(PlayTennis) - Entropy(PlayTennis, Temperature) \\
    & = 0.94 - 0.911 = 0.029 \\
    Gain(PlayTennis, Humidity) & \\
    & = Entropy(PlayTennis) - Entropy(PlayTennis, Humidity) \\
    & = 0.94 - 0.78835 = 0.15165 \\
    Gain(PlayTennis, Wind) & \\
    & = Entropy(PlayTennis) - Entropy(PlayTennis, Wind) \\
    & = 0.94 - 0.8922 = 0.0478 \\
  \end{split}
\end{equation*}

By looking at the leftover events now that we know Overcast has the most entropy and is the root we can see that when it is Sunny the catigory with the highest entropy is Humidity and similarly when it is rainy the highest entropy is Wind

\begin{tikzpicture}[x=.035\textwidth,y=.035\textwidth]
       \node[simple] (1) {Outlook};
       
      \draw[->] (1.center) ++(0, -10pt) -- +(-6,-2) node[simple] (2) {Sunny}; 
       \draw[->] (1.center) ++(0, -10pt) -- +(0,-2) node[simple] (3) {Overcast}; 
       \draw[->] (3.center) ++(0, -10pt) -- +(0,-2) node[simple] (5) {Play=Yes}; 
       \draw[->] (1.center) ++(0, -10pt) -- +(+6,-2) node[simple] (4) {Rainy}; 
       \draw[->] (2.center) ++(0, -10pt) -- +(0,-2) node[simple] (6) {Humidity}; 
       \draw[->] (6.center) ++(0, -10pt) -- +(-3,-2) node[simple] (8) {High Play=No}; 
       \draw[->] (6.center) ++(0, -10pt) -- +(+3,-2) node[simple] (9) {Weak Play=Yes}; 
       \draw[->] (4.center) ++(0, -10pt) -- +(0,-2) node[simple] (7) {Wind}; 
       \draw[->] (7.center) ++(0, -10pt) -- +(-3,-2) node[simple] (10) {Strong Play=No}; 
       \draw[->] (7.center) ++(0, -10pt) -- +(+3,-2) node[simple] (11) {Weak Play=Yes}; 
       
  \end{tikzpicture}

\section{Part 2}
build	the	decision	tree	by	using	information	content.

\end{document}

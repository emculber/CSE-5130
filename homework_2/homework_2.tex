\documentclass{report}
\usepackage[showframe=false]{geometry}
\usepackage{titlesec}
\usepackage{amsmath}
\usepackage{graphicx}
\usepackage{tikz}

\pagenumbering{gobble}

\geometry{tmargin=60pt,bmargin=90pt,lmargin=90pt,
rmargin=90pt}

\titleformat{\chapter}{\normalfont\huge}{\thechapter.}{20pt}{\huge}
\titlespacing*{\chapter} {0pt}{0pt}{10pt}

\definecolor{cadmiumgreen}{rgb}{0.0, 0.42, 0.24}

\begin{document}

\chapter{Question 1}

Consider  one auto  company that  receives  parts from  three suppliers;  assume  
50\% of  the parts from  supplier  1,  30\% from  supplier  2,  and 20\% from  supplier  3.  
The quality of  the parts could be  summarized  in  the following table based on  
historically  data.

\begin{center}
  \begin{tabular}{ | c | c | c | } 
    \hline
      & Percentage Good Parts & Percentage Bad parts \\ 
    \hline
    Supplier 1 & 98 & 2 \\ 
    \hline
    Supplier 2 & 95 & 5 \\ 
    \hline
    Supplier 3 & 92 & 8 \\ 
    \hline
  \end{tabular}
\end{center}

Question: A bad part  broke one of  the machines  (observed), what  is  the 
probability the part  came  from  supplier  1?

\hspace{1cm}

\textbf{ANSWER:} \\

Let $A_1$ denote Supplier 1, $A_2$ denote Supplier 2, and $A_3$ denote Supplier 3

\begin{equation} \label{eq3}
  \begin{split}
    P(A_1) & = 0.50 \\
    P(A_2) & = 0.30 \\
    P(A_3) & = 0.20 \\
  \end{split}
\end{equation}

Let G denote that a part is good and B denote that a part is bad. \\

\begin{equation} \label{eq3}
  \begin{split}
    P(G|A_1) & = 0.98 \\
    P(G|A_2) & = 0.95 \\
    P(G|A_3) & = 0.92 \\
  \end{split}
\end{equation}

\begin{equation} \label{eq3}
  \begin{split}
    P(B|A_1) & = 0.02 \\
    P(B|A_2) & = 0.05 \\
    P(B|A_3) & = 0.08 \\
  \end{split}
\end{equation}

\break

Here is the Probability Tree for Three-Suppliers

\hspace{1cm}

% Set the overall layout of the tree
\tikzstyle{level 1}=[level distance=3.5cm, sibling distance=3.5cm]
\tikzstyle{level 2}=[level distance=3.5cm, sibling distance=2cm]

% Define styles for bags and leafs
\tikzstyle{bag} = [circle, minimum width=3pt, fill, inner sep=0pt]
\tikzstyle{end} = [circle, minimum width=3pt, fill, inner sep=0pt]

% The sloped option gives rotated edge labels. Personally
% I find sloped labels a bit difficult to read. Remove the sloped options
% to get horizontal labels. 
\begin{tikzpicture}[grow=right, sloped]
  \node[bag] {$\circ$}
  child {
    node[bag] {$\circ$}
    child {
      node[end, label=right:
      {$P(A_3\cap B)=P(A_3)P(B|A_3)=0.016$}] {}
      edge from parent
      node[above] {$P(B|A_3)$}
    node[below]  {\textcolor{red}{$0.08$}}
    }
    child {
      node[end, label=right:
      {$P(A_3\cap G)=P(A_3)P(G|A_3)=0.184$}] {}
      edge from parent
      node[above] {$P(G|A_3)$}
      node[below]  {\textcolor{cadmiumgreen}{$0.92$}}
    }
    edge from parent 
    node[above] {$P(A_3)$}
    node[below]  {\textcolor{blue}{$0.20$}}
  } child {
    node[bag] {$\circ$}
    child {
      node[end, label=right:
      {$P(A_2\cap B)=P(A_2)P(B|A_2)=0.015$}] {}
      edge from parent
      node[above] {$P(B|A_2)$}
      node[below]  {\textcolor{red}{$0.05$}}
    }
    child {
      node[end, label=right:
      {$P(A_2\cap G)=P(A_2)P(G|A_2)=0.285$}] {}
      edge from parent
      node[above] {$P(G|A_2)$}
      node[below]  {\textcolor{cadmiumgreen}{$0.95$}}
    }
    edge from parent         
    node[above] {$P(A_2)$}
    node[below]  {\textcolor{blue}{$0.30$}}
  } child {
    node[bag] {$\circ$}
    child {
      node[end, label=right:
      {$P(A_1\cap B)=P(A_1)P(B|A_1)=0.010$}] {}
      edge from parent
      node[above] {$P(B|A_1)$}
      node[below]  {\textcolor{red}{$0.02$}}
    }
    child {
      node[end, label=right:
      {$P(A_1\cap G)=P(A_1)P(G|A_1)=0.490$}] {}
      edge from parent
      node[above] {$P(G|A_1)$}
      node[below]  {\textcolor{cadmiumgreen}{$0.98$}}
    }
    edge from parent         
    node[above] {$P(A_1)$}
    node[below]  {\textcolor{blue}{$0.50$}}
  };
\end{tikzpicture}

\hspace{1cm}

Law of Conditional Probability gives us the following equation

\begin{equation} \label{eq3}
  P(A_1|B) = \frac{P(A_1\cap B)}{P(B)}
\end{equation}

We know from are probability tree that

\begin{equation} \label{eq3}
P(A_1\cap B)=P(A_1)P(B|A_1)=0.010
\end{equation}

We also know that

\begin{equation} \label{eq3}
  P(B) = P(A_1\cap B) + P(A_2\cap B) + P(A_3\cap B) = P(A_1)P(B|A_1) + P(A_2)P(B|A_2) + P(A_3)P(B|A_3)
\end{equation}

Combining are equations togther we obtain Bayes' Theorem

\begin{equation}
  \begin{split}
    & 1 \leq k \leq n \\
    P(B_k|A) & = \frac{P(A|B_k)P(B_k)}{\sum_{i=1}^{n}P(A|B_i)P(B_i)}
  \end{split}
\end{equation}

\begin{equation}
  \begin{split}
    P(A_1|B) & = \frac{P(A_1)P(B|A_1)}{P(A_1)P(B|A_1)+P(A_2)P(B|A_2)+P(A_3)P(B|A_3)} \\
    P(A_2|B) & = \frac{P(A_2)P(B|A_2)}{P(A_1)P(B|A_1)+P(A_2)P(B|A_2)+P(A_3)P(B|A_3)} \\
    P(A_3|B) & = \frac{P(A_3)P(B|A_3)}{P(A_1)P(B|A_1)+P(A_2)P(B|A_2)+P(A_3)P(B|A_3)} \\
  \end{split}
\end{equation}

\break

Now all we have to do is plug in the numbers and solve the equations

\begin{equation}
  \begin{split}
  P(A_1|B) & = \frac{(0.50)(0.02)}{(0.50)(0.02)+(0.30)(0.05) +(0.20)(0.08)} = \frac{0.010}{0.041} = 0.2439024390 \\
  P(A_2|B) & = \frac{(0.30)(0.05)}{(0.50)(0.02)+(0.30)(0.05) +(0.20)(0.08)} = \frac{0.015}{0.041} = 0.3658536585 \\
  P(A_3|B) & = \frac{(0.20)(0.08)}{(0.50)(0.02)+(0.30)(0.05) +(0.20)(0.08)} = \frac{0.016}{0.041} = 0.3902439024 \\
  \end{split}
\end{equation}

Therefore the probability the part came from supplier 1: 0.243902439

\chapter{Question 2}

For the play  tennis  data  set shown below:

\begin{tabular}{ |c||c c c c | c | }
  \hline
  Day & Outlook & Temperature & Humidity & Wind & PlayTennis \\
  \hline
  D1  & Sunny    & Hot  & High   & Weak   & No \\
  D2  & Sunny    & Hot  & High   & Strong & No \\
  D3  & Overcast & Hot  & High   & Weak   & Yes \\
  D4  & Rain     & Mild & High   & Weak   & Yes \\
  D5  & Rain     & Cool & Normal & Weak   & Yes \\
  D6  & Rain     & Cool & Normal & Strong & No \\
  D7  & Overcast & Cool & Normal & Strong & Yes \\
  D8  & Sunny    & Mild & High   & Weak   & No \\
  D9  & Sunny    & Cool & Normal & Weak   & Yes \\
  D10 & Rain     & Mild & Normal & Weak   & Yes \\
  D11 & Sunny    & Mild & Normal & Strong & Yes \\
  D12 & Overcast & Mild & High   & Strong & Yes \\
  D13 & Overcast & Hot  & Normal & Weak   & Yes \\
  D14 & Rain     & Mild & High   & Strong & No \\
  \hline
\end{tabular}

\section{Part 1}

Please use	Naïve	Bayes	to	help	make	decision	on	playing	tennis	or	not	when
Outlook	is	rain,	Temperature	is	mild,	Humidity	is	normal and	Wind	is	weak.

\section{Part 2}

	The	humidity	value	could	be	continuous	practically.	In	the	above	data	set,	if
the	humidity	value	is	as	follows per	original	data	set	order:
\begin{equation*}
  \begin{split}
    Yes: & 65.7,	20.7,	5.1,	6.9,	4.8,	6.9,	8.7,	10.4,	15.3, \\
    No:	& 58.1,	66.4,	6.5,	10.5,	12.8 \\
  \end{split}
\end{equation*}
Please	use	Naïve	Bayes	to	help	make	decision	on	playing	tennis	or	not	when	the
Outlook	is	Overcast,	Temperature	is	mild,	Humidity	is	8.8,	and	Wind	is	weak.

\end{document}
